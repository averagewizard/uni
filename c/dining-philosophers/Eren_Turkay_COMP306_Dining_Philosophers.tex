% vim: tw=72:spell:
\documentclass[a4paper,12pt]{article}

\usepackage[utf8]{inputenc}
\usepackage{setspace}
\usepackage{fancyhdr}
\usepackage{indentfirst} % indent after section, subsection etc
\usepackage{capt-of} % to be able to write captions

% Use double space
%\linespread{2}

% Use fancy header
\pagestyle{fancy}
\rhead{Eren Türkay}
\lhead{Dining Philosophers Problem}

\author{Eren TÜRKAY\\
    \small Department of Computer Science,\\
    \small Istanbul Bilgi University,\\
    \small Istanbul,\\
    \small Turkey\\
    \texttt{\small turkay.eren@gmail.com}}

\title{COMP306 - Dining Philosophers Problem}
\date{\today}

\begin{document}
\maketitle
\newpage

\section{Introduction} % (fold)
\label{sec:Introduction}

\emph{Dining philosophers} is an example problem used to describe the
problems, and techniques for solving these problems in a concurrent
programming environment. The problem was originally formulated in 1965
by Edsger W. Dijsktra. It was proposed as an example to
illustrate concurrency and synchronization problems for his students.

The problem is stated with \emph{n} number of philosophers sitting
around the dinner table. On the table, there are \emph{n} number of
forks and \emph{n} number of tables. Usually, the problem is stated with
\emph{n=5}. Accordingly to Siblerschatz, the problem is defined as
follows:

\begin{quote}
Consider five philosophers who spend their lives thinking and eating.
The philosophers share a circular table surrounded by five chairs, each
belonging to one philosopher. In the center of the table is a bowl of
rice, and the table is laid with five single chopsticks.  When a
philosopher thinks, she does not interact with her colleagues.  From
time to time, a philosopher gets hungry and tries to pick up the two
chopsticks that are closest to her (the chopsticks that are between her
and her left and right neighbors). A philosopher may pick up only one
chopstick at a time. Obviously, she cannot pick up a chopstick that is
already in the hand of a neighbor. When a hungry philosopher has both
her chopsticks at the same time, she eats without releasing her
chopsticks. When she is finished eating, she puts down both of her
chopsticks and starts thinking again. \cite{siblerschatz}
\end{quote}

In Dining Philosophers problem, chopsticks represent the shared
resource, and the philosophers represent the processes that want to
access these resources. The problem is important because it allows us to
intuitively understand concurrency problems.

% section Introduction (end)

\section{Main Problem} % (fold)
\label{sec:Main Problem}

The main problem in \emph{Dining Philosophers} is either that they starve to
death or that they hold the stick at the same time so they cannot think
or eat.

Imagine that each philosopher sits on the table at the same time and
holds one of the sticks. In this case, all chopsticks are occupied and
philosophers wait each other for another stick. No philosopher can eat
or think. Hence, a deadlock occurs.

When two philosophers are fast thinkers and eaters, they can occupy 4
chopsticks so fast that other philosophers have no chance of getting
chopsticks. This leads to "starvation". It is different from a deadlock
in that 2 philosophers think\&eat, others starve.

% section Main Problem (end)

\section{Techniques For Solving} % (fold)
\label{sec:Techniques For Solving}

There are different ways of solving Dining Philosophers Problem. In this
section, possible solutions to the problem are given. In each section,
the way, and the problems that the solution posses are explained.

\subsection{Semaphores} % (fold)
\label{sub:Semaphores}

Using semaphore is one of the simple solutions to the problem. Each
chopstick is represented with a semaphore. 

A philosopher tries to grab a chopstick by executing a wait() operation
on that semaphore; she releases her chopsticks by executing the signal()
operation on the appropriate semaphores. Thus, the shared data are

\begin{quote}
\begin{verbatim}
semaphore chopstick[5];
\end{verbatim}
\end{quote}

where all the elements of chopstick are initialized to 1. So, the
algorithm when using semaphores is:

\begin{enumerate}
    \item Lock left chopstick
    \item Lock right chopstick
    \item Eat
    \item Release left chopstick
    \item Release right chopstick
    \item Think
\end{enumerate}

Although this solution guarantees that no two neighbors are eating
simultaneously, it nevertheless must be rejected because it could create
a deadlock. Suppose that all five philosophers become hungry
simultaneously and each grabs her left chopstick. All the elements of
chopstick will now be equal to 0. When each philosopher tries to grab
her right chopstick, she will be delayed forever. \cite{siblerschatz}

There are a few ways for encountering the deadlock problem presented by
using semaphores:

\begin{enumerate}
    \item Allow at most four philosophers to be sitting simultaneously
        at the table

    \item Allow a philosopher to pick up her chopstick only if both
        chopsticks are available. (to do this, she must pick them up in
        a critical section)

    \item Use an asymmetric solution; that is, an odd philosopher picks
        up first her left chopstick and then her right chopstick,
        whereas an even philosopher picks up her right chopstick and
        then her left chopstick.
\end{enumerate}

% subsection Semaphores (end)

\subsection{Monitors} % (fold)
\label{sub:Monitors}

This method of solving \emph{Dining Philosophers Problem} arises from
the fact that using semaphores is prone to programming errors. When a
programmer incorrectly use semaphores, deadlock or starvation could
easily occur.  With monitors, synchronization and concurrency are
achieved through more abstract classes which are not error prone.

A monitor is an \emph{ADT - Abstract Data Type} which encapsulates
private data with public methods to operate on that data. A monitor type
is an ADT that presents a set of programmer-defined operations that are
provided mutual exclusion within the monitor. The monitor type also
contains the declaration of variables whose values define the state of
an instance of that type, along with the bodies of procedures or
functions that operate on those variables.

The monitor construct ensures that only one process at a time is active
within the monitor. Consequently, the programmer does not need to code
this synchronization constraint explicitly. This model of monitoring is
present in most of the Object Oriented Languages including \emph{Java,
Smalltalk, C++, etc}.

The solution to \emph{Dining Philosophers Problem} using monitors
imposes the restriction that a philosopher may pick up her chopsticks
only if both of them are available. For this purpose, a new data
structure is created:

\begin{quote}
\begin{verbatim}
enum {THINKING, HUNGRY, EATING} state[5];
\end{verbatim}
\end{quote}

Philosopher i can set the variable \texttt{state[i] = EATING} only if
her two neighbors are not eating: \texttt{(state[(i+4) \% 5] !=
EATING) AND (state[(i+1) \% 5] != EATING)}

% subsection Monitors (end)

% section Techniques For Solving (end)

\section{My Solution} % (fold)
\label{sec:My Solution}

My solution involves using semaphores, and checking if both chopsticks
are available to avoid deadlock. The code is self-documented, which can
be easily understood.

\subsection{Downside} % (fold)
\label{sub:Downside}

Currently, the code does not have a deadlock as a thread starts eating
only if both chopsticks are available. However, the code is prone to
\emph{starvation}. If two threads are fast thinkers and fast eaters,
other threads will not be able to acquire a chopstick, and resources
will be only available to fast thinking\&eating threads.

There are ways to overcome this problem. One way is to count the number
of eating time and compare this time with other threads. If one thread
is eating too many times, it is not given an opportunity to get
chopsticks. Chopsticks are given to other threads. The program controls
this eating time and balance it between threads.

% subsection Downside (end)

% section My Solution (end)


\begin{thebibliography}{9}
    \bibitem{siblerschatz}
        Silberschatz, Abraham (2011)
        Operating system concepts essentials.
        \emph{Process Synchronization} (p. 234).
        Hoboken, NJ: John Wiley \& Sons Inc.
\end{thebibliography}



\end{document}
